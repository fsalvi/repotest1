% This is the UK TeX FAQ
%
\documentclass{faq}
%
\setcounter{errorcontextlines}{999}
%
% read the first two definitions of faqbody.tex for the file version
% and date
\afterassignment\endinput\input gather-faqbody
%
\typeout{The UK TeX FAQ, v\faqfileversion, date \faqfiledate}
%
% let's not be too fussy while we're developing...
\hfuzz10pt
\emergencystretch10pt
%
\begin{document}
\title{The \acro{UK} \TeX{} \acro{FAQ}\\
  Your \ifpdf\ref*{lastquestion}\else\protect\ref{lastquestion}\fi
  \ Questions Answered\\
       version \faqfileversion, date \faqfiledate}
\author{} 
\maketitle

\centerline{\textsc{Note}}

\begin{quotation}\small
  This document is an updated and extended version of the \acro{FAQ} article
  that was published as the December 1994 and 1995, and March 1999
  editions of the \acro{UK}\,\acro{TUG} magazine \BV{} (which weren't
  formatted like this).

  The article is also available via the World Wide Web.
\end{quotation}

\ifsinglecolumn
  \tableofcontents
\else
  \begin{multicols}{2}
  \tableofcontents
  \end{multicols}
\fi

\Dings

% label list for later processing
\labellist

%%%%%%%%%%%%%%%%%%%%%%%%%%%%%%%%%%%%%%%%%%%%%%%%%%%%%%%%%%%%%%%%%
% load the CTAN references if necessary
\input{dirctan}
\input{filectan}
%%%%%%%%%%%%%%%%%%%%%%%%%%%%%%%%%%%%%%%%%%%%%%%%%%%%%%%%%%%%%%%%%

\ifsinglecolumn
  \def\faqfileversion{3.27}    \def\faqfiledate{2013-06-07}
%
% The above line defines the file version and date, and must remain
% the first line with any `assignment' in the file, or things will
% blow up in a stupid fashion
%
% get lists of CTAN labels
%
% configuration for the lists, if we're going to need to generate urls
% for the files
\InputIfFileExists{archive.cfg}{}{}
%
% ... directories
\input{dirctan}
%
% ... files
\input{filectan}

% facilitate auto-processing of this stuff; this line is clunkily
% detected (and ignored) in the build-faqbody script
\let\faqinput\input
%\def\faqinput#1{\message{*** inputting #1; group level
%  \the\currentgrouplevel}\input{#1}%
%  \message{*** out of #1; group level \the\currentgrouplevel}}

% the stuff to print
\faqinput{faq-intro}               % introduction
\faqinput{faq-backgrnd}            % background
\faqinput{faq-docs}                % docs
\faqinput{faq-bits+pieces}         % bits and pieces
\faqinput{faq-getit}               % getting software
\faqinput{faq-texsys}              % TeX systems
\faqinput{faq-dvi}                 % DVI drivers and previewers
\faqinput{faq-support}             % support stuff
\faqinput{faq-litprog}             % programming for literates
\faqinput{faq-fmt-conv}            % format conversions
\faqinput{faq-install}             % installation and so on
\faqinput{faq-fonts}               % fonts and what to do
\faqinput{faq-hyp+pdf}             % hyper-foodle and pdf
\faqinput{faq-graphics}            % graphics
\faqinput{faq-biblio}              % bib stuff
\faqinput{faq-adj-types}           % adjusting typesetting
\faqinput{faq-lab-ref}             % labels and references
\faqinput{faq-how-do-i}            % how to do things
\faqinput{faq-symbols}             % symbols, their natural history and use
\faqinput{faq-mac-prog}            % macro programming
\faqinput{faq-t-g-wr}              % things going wrong
\faqinput{faq-wdidt}               % why does it do that
%*****************************************quote environments up to here
\faqinput{faq-jot-err}             % joy (hem hem) of tex errors
\faqinput{faq-projects}            % current projects
%
% This is the last section, and is to remain the last section...
\faqinput{faq-the-end}             % wrapping it all up

% \end{document} is in calling file (e.g., newfaq.tex)



\else
  \begin{multicols}{2}
  \def\faqfileversion{3.27}    \def\faqfiledate{2013-06-07}
%
% The above line defines the file version and date, and must remain
% the first line with any `assignment' in the file, or things will
% blow up in a stupid fashion
%
% get lists of CTAN labels
%
% configuration for the lists, if we're going to need to generate urls
% for the files
\InputIfFileExists{archive.cfg}{}{}
%
% ... directories
\input{dirctan}
%
% ... files
\input{filectan}

% facilitate auto-processing of this stuff; this line is clunkily
% detected (and ignored) in the build-faqbody script
\let\faqinput\input
%\def\faqinput#1{\message{*** inputting #1; group level
%  \the\currentgrouplevel}\input{#1}%
%  \message{*** out of #1; group level \the\currentgrouplevel}}

% the stuff to print
\faqinput{faq-intro}               % introduction
\faqinput{faq-backgrnd}            % background
\faqinput{faq-docs}                % docs
\faqinput{faq-bits+pieces}         % bits and pieces
\faqinput{faq-getit}               % getting software
\faqinput{faq-texsys}              % TeX systems
\faqinput{faq-dvi}                 % DVI drivers and previewers
\faqinput{faq-support}             % support stuff
\faqinput{faq-litprog}             % programming for literates
\faqinput{faq-fmt-conv}            % format conversions
\faqinput{faq-install}             % installation and so on
\faqinput{faq-fonts}               % fonts and what to do
\faqinput{faq-hyp+pdf}             % hyper-foodle and pdf
\faqinput{faq-graphics}            % graphics
\faqinput{faq-biblio}              % bib stuff
\faqinput{faq-adj-types}           % adjusting typesetting
\faqinput{faq-lab-ref}             % labels and references
\faqinput{faq-how-do-i}            % how to do things
\faqinput{faq-symbols}             % symbols, their natural history and use
\faqinput{faq-mac-prog}            % macro programming
\faqinput{faq-t-g-wr}              % things going wrong
\faqinput{faq-wdidt}               % why does it do that
%*****************************************quote environments up to here
\faqinput{faq-jot-err}             % joy (hem hem) of tex errors
\faqinput{faq-projects}            % current projects
%
% This is the last section, and is to remain the last section...
\faqinput{faq-the-end}             % wrapping it all up

% \end{document} is in calling file (e.g., newfaq.tex)



  \end{multicols}
\fi

\typeout{*** That makes \thequestion\space questions ***}
\end{document}
