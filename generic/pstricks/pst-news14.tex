%% $Id: pst-news13.tex 856 2013-12-09 10:34:40Z herbert $
\documentclass[11pt,english,BCOR10mm,DIV12,bibliography=totoc,parskip=false,smallheadings
    headexclude,footexclude,oneside]{pst-doc}
\listfiles
\let\Lfile\LFile
\usepackage[utf8]{inputenc}
\usepackage{pst-node}
\let\pstnodeFV\fileversion
\let\pstnodeFD\filedate
\usepackage{pst-plot}
\usepackage{xkvview}
\renewcommand\bgImage{\psscalebox{15}{\color{blue!20}2014}}
\def\textat{\char064}
\lstset{explpreset={pos=l,width=-99pt,overhang=0pt,hsep=\columnsep,vsep=\bigskipamount,rframe={}},
    escapechar=?}
\begin{document}

%\psset{PstDebug=1}
\title{\texttt{News -- 2014}\\ \Large new macros and bugfixes for the
basic package \nxLFile{pstricks}}
\author{Herbert Voß}
\date{\today}

\maketitle

\clearpage
\tableofcontents

\clearpage
\part{\texttt{pstricks} -- package}

%--------------------------------------------------------------------------------------
%\section{\texttt{pstricks.sty}}
%--------------------------------------------------------------------------------------


%--------------------------------------------------------------------------------------
\section{\texttt{pstricks.tex} (\pstricksFV -- \pstricksFD)}
%--------------------------------------------------------------------------------------
\section{Opacity}
The keyword \Lkeyword{strokeopacity} is now also valid for \Lcs{psdot}, \Lcs{psdots},
and the \Lkeyword{linestyle}/\Lkeyword{plotstyle}=\Lkeyval{dots}.


\subsection{PostScript notation for numbers}
Optional arguments which expects a real number can now have a preceeding ! character for
a PostScript notation which is directly passed to PostScript. The user has take care that
such a number isn't use before in another \TeX\ macro. In such a case it gives an error.

\begin{LTXexample}[width=5cm]
\pstVerb{ 1234321 srand }
\begin{pspicture}[showgrid](-2,-2)(2,2)
\psframe*[linecolor=blue,opacity=!Rand](2,2)
\psframe*[linecolor=red,opacity=!Rand](-1,-1)(1,1)
\psframe*[linecolor=green,opacity=!Rand](-2,-2)(0,0)
\end{pspicture}
\end{LTXexample}


\subsection{Fillstyle \texttt{eofill}}

It is an experimental fillstyle. PostScript knows only the \Lkeyval{eofill} and the other way round
needs some tricky internal commands and may not work in all cases.

\begin{LTXexample}[pos=t]
\begin{pspicture}[linewidth=2pt](12,4)    
\pscustom[linestyle=none,fillstyle=eofill,fillcolor=blue!40]{%
   \psellipse(4,2)(2,2)\psellipse(2,2)(2,2)}
\psellipse[linecolor=red](4,2)(2,2)
\psellipse[linecolor=green](2,2)(2,2)
%
\pscustom[linestyle=none,fillstyle=oefill,fillcolor=blue!40]{%
   \psellipse(10,2)(2,2)\psellipse(8,2)(2,2)}
\psellipse[linecolor=red](10,2)(2,2)
\psellipse[linecolor=green](8,2)(2,2)
\end{pspicture}
\end{LTXexample}



\subsection{New macro \nxLcs{psRing}}

\begin{BDef}
\LcsStar{psRing}\OptArgs\Largr{\CAny}\OptArg{start,end}\Largb{Inner Radius}\Largb{Outer Radius}
\end{BDef}


\begin{LTXexample}[width=5cm]
\begin{pspicture}[showgrid](4,4)
  \psRing(2,2){0.3}{0.8}
  \psRing*[opacity=0.5](2,2){1}{2}
\psdot(2,2)
\end{pspicture}
\end{LTXexample}


\begin{LTXexample}[width=5cm]
\begin{pspicture}[showgrid](4,4)
  \psRing[linecolor=red](2,2)[30,60]{1}{2}
  \psRing[opacity=0.5,fillstyle=solid,
    fillcolor=red](2,2)[60,30]{1}{2}
\psdot(2,2)
\end{pspicture}
\end{LTXexample}


\clearpage
\nocite{*}
\bibliographystyle{plain}
\bibliography{PSTricks}

\printindex


\end{document}


