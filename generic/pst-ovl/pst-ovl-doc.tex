%% $Id: pst-func-doc.tex 599 2011-11-03 19:38:28Z herbert $
\documentclass[11pt,english,BCOR10mm,DIV12,bibliography=totoc,parskip=false,
   smallheadings, headexclude,footexclude,oneside]{pst-doc}
\usepackage[utf8]{inputenc}
\usepackage{pst-ovl}
\let\pstOVLFV\fileversion
\renewcommand\bgImage{}

\lstset{language=PSTricks,
    morekeywords={psPrintValue},basicstyle=\footnotesize\ttfamily}
%
\begin{document}

\title{\texttt{pst-ovl}}
\subtitle{Helper functions for overlays; v.\pstOVLFV}
\author{Herbert Vo\ss}
\docauthor{}
\date{\today}
\maketitle

\tableofcontents
\psset{unit=1cm}

\section{Overlays}

Overlays are mainly of interest for making slides, and the overlay macros
described in this section are mainly of interest to \TeX{} macro writers who
want to implement overlays in a slide macro package.  For example, the
\LPack{seminar} package, a \LaTeX{} style for notes and slides, uses PSTricks to
implement overlays.

Overlays are made by creating an "`\Lcs{hbox}"' and then outputting the box several
times, printing different material in the box each time. The box is created by
the commands
\begin{lstlisting}
  \Lcs{overlaybox} < stuff >\Lcs{endoverlaybox}
\end{lstlisting}
\LaTeX{} users can instead write:
\begin{lstlisting}
  \begin{overlaybox} <stuff> \end{overlaybox}
\end{lstlisting}

The material for overlay <string> should go within the scope of the command

  \Lcs{psoverlay}\Largb{string}

<string> can be any string, after expansion. Anything not in the scope of any
\Lcs{psoverlay} command goes on overlay "`main"', and material within the scope of
\Lcs{psoverlay}\Largb{all} goes on all the overlays. \Lcs{psoverlay}
commands can be nested and can be used in math mode.

The command

\Lcs{putoverlaybox}\Largb{string}

then prints overlay <string>.

Here is an example:

\begin{LTXexample}[pos=t]
\overlaybox
\psoverlay{all}
\psframebox[framearc=.15,linewidth=1.5pt]{%
  \psoverlay{main}
  \parbox{3.5cm}{\raggedright
    Foam Cups Damage Environment {\psoverlay{one} Less than
    Paper Cups,} Study Says.}}
 \endoverlaybox

\putoverlaybox{main} \hspace{.5in} \putoverlaybox{one}
\end{LTXexample}



\clearpage
\section{List of all optional arguments for \texttt{pst-ovl}}

\xkvview{family=pst-ovl,columns={key,type,default}}




\bgroup
\raggedright
\nocite{*}
\bibliographystyle{plain}
\bibliography{pst-ovl-doc}
\egroup

\printindex



\end{document}


