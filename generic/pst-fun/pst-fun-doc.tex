%% $Id: pst-thick-doc.tex 170 2009-12-08 21:30:53Z herbert $
\documentclass[11pt,english,BCOR10mm,DIV12,bibliography=totoc,parskip=false,smallheadings
    headexclude,footexclude,oneside]{pst-doc}
\usepackage[utf8]{inputenc}
\usepackage{pstricks-add,pst-fun}
\let\pstFV\fileversion

\def\bgImage{{%\psset{unit=0.5}
\begin{pspicture}(0,-1.3)(10,3)
  \psBird[Branch] \rput{-20}(4,1.8){\psBird}
\end{pspicture}
}}

\lstset{explpreset={pos=t,width=-99pt,overhang=0pt,hsep=\columnsep,vsep=\bigskipamount,rframe={}},style=code}
\begin{document}

\title{\texttt{pst-fun}\\funny macros\\
		  \small v.\pstFV}
\subtitle{A PSTricks package for drawing funny objects}
\author{Manuel Luque\\Herbert Vo\ss} 
\docauthor{Herbert Vo\ss}
\date{\today}
\maketitle

\clearpage

%\begin{abstract}
%\end{abstract}

\tableofcontents

\clearpage

\section{The macros}
\begin{LTXexample}
\begin{pspicture}[showgrid](-5,-4)(4,8) % needs pstricks-add
  \psBill
  \psHomothetie[linecolor=blue](4,-3){2}{\psBill}
  \psdots[dotsize=3pt,linecolor=red](4,-3)
  \pstVerb{ /m -3 -0.85 sub 4 0.6 sub div def }
  \psplot[linestyle=dashed,linecolor=red]{-5}{4}{ m x mul m 4 mul sub 3 sub }
  \psHomothetie[linecolor=green](4,-3){0.5}{\psBill}
  \psHomothetie[linecolor=magenta](4,-3){-0.25}{\psBill}
\end{pspicture}
\end{LTXexample}
\xLcs{psBill}
%$


\begin{LTXexample}
\begin{pspicture}[showgrid](0,-0.5)(12,4)
 \psFish
 \rput(6,0){\psFish[fillstyle=slope]}
\end{pspicture}
\end{LTXexample}
\xLcs{psFish}


\begin{LTXexample}
\begin{pspicture}[showgrid](-2,-2.4)(6,2)
  \psLouisXIII
  \rput(4,0){\psLouisXIII[linecolor=red]}
\end{pspicture}
\end{LTXexample}
\xLcs{psLouisXIII}


\begin{LTXexample}
\begin{pspicture}[showgrid](0,-0.4)(11,6)
  \psPulpo \rput(10,0.5){\psscalebox{0.15}{\psPulpo[fillcolor=yellow,fillstyle=solid]}}
\end{pspicture}
\end{LTXexample}
\xLcs{psPulpo}


\begin{LTXexample}
\begin{pspicture}[showgrid](0,-1.2)(12,3)
  \psBird \rput(4,0){\psBird} \rput{-60}(8,2){\psBird}
\end{pspicture}
\end{LTXexample}
\xLcs{psBird}\xLkeyword{Branch}

\begin{LTXexample}
\begin{pspicture}[showgrid](0,-1.3)(10,3)
  \psBird[Branch] \rput{-20}(4,1.8){\psBird}
\end{pspicture}
\end{LTXexample}
\xLcs{psBird}\xLkeyword{Branch}

\begin{LTXexample}
\begin{pspicture}[showgrid](0,-0.3)(8,10)
  \psLuke
  \rput(6.5,1){\psscalebox{0.15}{\psLuke}}
\end{pspicture}
\end{LTXexample}
\xLcs{psLuke}

\begin{LTXexample}
\begin{pspicture}[showgrid](-5,-5)(5,5)
  \psAnt
  \rput(-3,3){\psAnt[fillcolor=red!50]}
  \rput{30}(3,-3){\psAnt[fillcolor=blue!50]}
  \rput{-60}(-3,-3){\psAnt[fillcolor=blue!20]}
  \rput(2.5,3){\psscalebox{0.15}{\psAnt}}
\end{pspicture}
\end{LTXexample}
\xLcs{psAnt}

\begin{LTXexample}
\begin{pspicture}[showgrid](6,7)
  \psParrot{1}
  \psParrot{0.2}\rput(4,5){\psParrot{0.2}}
\end{pspicture}
\end{LTXexample}
\xLcs{psParrot}

\begin{LTXexample}
\begin{pspicture}[showgrid](8,7)
  \psKangaroo{1}
  \multido{\iA=2+1}{5}{\rput[lb](1,\iA){\psKangaroo[fillcolor=red]{1}}}
  \rput(4,0){\psKangaroo[fillcolor=red!30]{5}}
  \rput(5,1){\psKangaroo[fillcolor=blue!30,opacity=0.5]{5}}
\end{pspicture}
\end{LTXexample}
\xLcs{psKangaroo}

\begin{LTXexample}
\begin{pspicture}(-1,-4)(8,4)
  \psPig(0,0)\psPig[fillcolor=blue!40,noseColor=purple,
    eyeColor=red,linewidth=4pt,unit=2](4,0)
\end{pspicture}
\end{LTXexample}
\xLcs{psPig}\xLkeyword{eyeColor}\xLkeyword{noseColor}

\clearpage
\section{List of all optional arguments for \texttt{pst-fun}}

\xkvview{family=pst-fun,columns={key,type,default}}


\bgroup
\raggedright
\nocite{*}
\bibliographystyle{plain}
\bibliography{pst-fun-doc}
\egroup


\printindex
\end{document}


